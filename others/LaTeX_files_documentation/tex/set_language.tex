%!TEX root = main.tex
\chapter{El lenguaje \Set}

	\section{Introducción al lenguaje \Set}

		Cuando estudiamos un nuevo lenguaje natural como el inglés, el alemán o el chino, empezamos aprendiendo el alfabeto de símbolos usados en el lenguaje, a continuación formamos palabras con esos símbolos. Lo siguiente es aprender la manera correcta en la cual se deben colocar las palabras para formar frases y el significado de dichas combinaciones de palabras. La misma idea se aplica al lenguaje de programación de alto nivel \Set\footnote
		{
			Un lenguaje de programación de «alto nivel» se caracteriza por expresar los algoritmos de una manera adecuada a la capacidad cognitiva humana, en lugar de la capacidad ejecutora de la máquina de las máquinas. Para ello se utilizan términos o palabras propias del lenguaje natural o del lenguaje Matemático.
		}. %
		La característica principal de este lenguaje es que permite representar los algoritmos de forma muy parecida a como se hace en los lenguajes de especificación de algoritmos ({\sc lea}); es decir, mediante uso de operandos, términos y expresiones que son propias del lenguaje matemático. Para ello se permiten algunos símbolos del juego de \emph{caracteres unicode} con los que representar de forma explícita 
		\begin{itemize}
			\item tipos de datos como conjuntos numéricos matemáticos (ℕ, ℤ, ℚ), 

			\item declaraciones de variables con el operador de pertenencia de conjuntos ∈,

			\item asignaciones de valores a variables con el símbolo ← o 

			\item declaraciones de funciones con los símbolos × (para separar los parámetros) o →, entre otros.
		\end{itemize}

		Este lenguaje carece de los conceptos de referencia, puntero, reserva y liberación de memoria, y dado se permite el uso de notación matemática, es el lenguaje de alto nivel más puro que existe.



	\section{Un programa en \Set}

		Un programa fuente en lenguaje \Set está compuesto por
		\begin{itemize}
			\item un fichero denominado \emph{principal} que puede tener cualquier nombre con extensión \cod{.mset}\footnote
			{
				La \cod{m} que aparece en la extensión \cod{.mset} es la primera letra de la palabra inglesa «main» que significa «principal». 
			},
			y por

			\item cero o varios \emph{módulos} que pueden tener cualquier nombre y cuya extensión es
			\begin{itemize}
				\item \cod{.hset} para los módulos de cabecera\footnote
				{
					La \cod{h} que aparece en la extensión \cod{.hset} es la primera letra de la palabra inglesa «header» que significa «cabecera». 
				}
				y 

				\item \cod{.iset} para los módulos de implementación\footnote
				{
					La \cod{i} que aparece en la extensión \cod{.iset} es la primera letra de la palabra inglesa «implementation» que significa «implementar». 
				}.
			\end{itemize}
		\end{itemize}

		Para compilar un programa fuente en lenguaje \Set debe ejecutar en su terminal el comando \cod{cset}, seguido de la ruta donde se encuentra el programa fuente principal (el que tiene la extensión \cod{.mset}) seguido de su nombre, con o sin extensión. Para poder ejecutar el comando de compilación \cod{cset} debe haber instalado previamente el compilador \Set en su sistema (consulte el apéndice~\ref{app:installation_set_compiler}).

		Por ejemplo, abra su editor de texto plano preferido y escriba el siguiente programa fuente:
		\lstinputlisting[language=set,caption={Mensaje «Hola mundo» \getlstname}]{../listings/set/examples/hello_world.mset}
		
		Guarde el programa fuente con el nombre que prefiera y añádale la extensión \cod{.mset} justo a continuación del nombre. Suponiendo que se ha guardado en el escritorio como \cod{hello\_world.set}, la compilación en diferentes sistemas se hace de la siguiente forma:
		\begin{itemize}
			\item %En Linux y en OS X escribiríamos: \cod{cset \textasciitilde/Desktop/hola_mundo.set}

			\item %En Windows escribiríamos: \cod{cset \%HOMEPATH\%\barInv Desktop\barInv hola_mundo.set}
		\end{itemize}

		Una vez compilado obtendrá un programa ejecutable que puede ejecutar de la siguiente forma:
		\begin{itemize}
			\item En linux y en OS X:

			\item En Windows:
		\end{itemize}

		Como se puede observar, la ejecución muestra en la terminal de su sistema el mensaje «\cod{Hola mundo}».

		Las funciones solo se pueden declarar e implementar en los módulos, de hecho todo el código que aparezca en un módulo debe estar dentro de una función. Con esto se fuerza al programador, desde el primer momento,  



	\section{Elementos léxicos}

		En las siguientes subsecciones se discuten los elementos léxicos del lenguaje \Set: comentarios, identificadores, palabras reservadas, números, caracteres y cadenas de caracteres.



		\subsection{Comentarios}

			Los comentarios son descripciones o aclaraciones del código en lenguaje natural que aparecen escritas junto a este y que ayudan a entender su estructura y su lógica (su significado). En el lenguaje \Set se permiten los siguientes dos tipos de comentarios:
			\begin{itemize}
				\item Comentarios monolínea. Se añaden escribiendo un punto y coma (\sbr{\cod{;}}) seguido por el texto del comentario como se muestra en el siguiente código:
				\lstinputlisting[language=set,caption={Comentario monolínea \getlstname}]{../listings/set/examples/singleline_comments.mset}

				\item Comentarios multilínea. Comienzan con los caracteres de apertura de comentario punto y coma guion (\sbr{\cod{;-}}), sin ningún espacio entre ambos caracteres, y se extienden hasta los caracteres de cierre guion punto y coma (\sbr{\cod{-;}}), de nuevo, sin ningún espacio entre ambos. Los caracteres de apertura y cierre pueden estar en diferentes lineas o sobre la misma linea como se muestra en el siguiente código:
				\lstinputlisting[language=set,caption={Comentario multilínea \getlstname}]{../listings/set/examples/multiline_comments.mset}
			\end{itemize}

			

	\section{Elementos sintácticos}



		\subsection{Declaración de funciones}

			Como ya se ha dicho en secciones anteriores

























